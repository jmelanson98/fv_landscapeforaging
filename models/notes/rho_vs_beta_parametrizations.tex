\documentclass[12pt]{article}
%\usepackage{iftex}
%\usepackage[pdftex]{graphicx}
%\usepackage{hyperref}
%\usepackage{color}
\usepackage{amsmath}
%\usepackage{amssymb}
%\usepackage{verbatim}
%\usepackage{lscape}
%\usepackage{textcomp}
%\usepackage{xr}
\setlength{\parindent}{0pt}
\setlength{\parskip}{1em}





\begin{document}
\section{$\beta$ and $\rho$ parametrizations}

There are two ways which I have tried parametrizing distance decay functions in my models. The first, used in the Pope paper, relies on a multiplicative $\beta$ term to scale the rate of distance decay. E.g.,

\[ln(\lambda) = -\beta |x_1 - x_2|\]

$\beta$ in this case is a small positive real. In my simulations is receives values from 0.01 - 0.1 although of course this would vary based on the units of distance (e.g., meters, kilometers, etc.). I assign it a prior,

\[
\beta \sim \mathcal{N}(0, 0.1)
\]

and it is restricted to positive values in the \texttt{parameters} block.

An alternative parametrization, following from Mike's chapter on the exponentiated quadratic function, is to use a divisional length scale $\rho$, such that

\[ln(\lambda_{ik}) = -\frac{1}{2}\left(\frac{|x_i - x_k|}{\rho} - \theta floralquality_k + \mu + \epsilon_k + \zeta_c\right)^2\]

\[ \rho \sim \mathcal{N}(100, 50)\]
\[ \theta \sim \mathcal{N}(0, 1)\]
\[ \mu \sim \mathcal{N}(0, 1)\]
\[ \epsilon_k \sim \mathcal{N}(0, \sigma)\]
\[ \zeta_c \sim \mathcal{N}(0, \tau)\]
\[ \sigma \sim \mathcal{N}(0, 1)\]
\[ \tau \sim \mathcal{N}(0, 1)\]

where $\rho$ (at least in my case) is a large positive real. For me, realistic values would be on the range 50-150. I initially gave it a prior,

\[\ \rho \sim \mathcal{N}(100, 50)\]

but then changed to a lognormal,

\[ \rho_{raw} \sim \mathcal{N}(ln(100), 0.5) \]
\[ \rho = exp(\rho_{raw}) \]



which is probably a bit too wide, to be fair.



\end{document}
