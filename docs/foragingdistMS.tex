\documentclass[12pt]{article}
\usepackage{iftex}
\usepackage[pdftex]{graphicx}
\usepackage{hyperref}
\usepackage{natbib}
\usepackage{color}
\usepackage{amsmath}
\usepackage{amssymb}
\usepackage[export]{adjustbox}
\usepackage{verbatim}
\usepackage{mathpazo}
\usepackage{setspace}
\usepackage{multirow}
\usepackage{fullpage}
\usepackage{lscape}
\usepackage{fancyhdr}
\usepackage[normalem]{ulem}
\usepackage{rotating}
\usepackage{chngcntr}
\usepackage[parfill]{parskip}
\usepackage[tiny,compact]{titlesec}
\usepackage{longtable}
\usepackage{textcomp}
\usepackage{xr}
%\externaldocument{supplementary_methods}

\newcommand{\flagged}[1] {
  \textcolor{blue}{#1}
}

\hypersetup{colorlinks=true, linkcolor=black, citecolor=black}
\RequirePackage{lineno}

\def\title{Insert title}

\def\author{Jenna B. Melanson$^{1*}$, Tyler T. Kelly$^{2}$, Natalia Clermont$^{2}$, \\
Jonathan B. Koch$^{3}$, Claire Kremen$^{1,2}$}

\def\keywords{keyword1, keyword2, keyword3}

\def\extras{
\begin{itemize}
  \item Type of article: 
  \item Abstract word count: 
  \item Word count:
  \item Number of figures and tables: 
  \item Number of references: 
  \item Author contributions: 
\end{itemize}

}
\clearpage
\def\affiliation{
  \begin{enumerate}
  \item Biodiversity Research Center\\
  Department of Zoology \\
  University of British Columbia \\
  2212 Main Mall \\
  Vancouver, BC, Canada, V6T 1Z4
  \item Institute for Resources, Environment, and Sustainability \\
  University of British Columbia \\
  2202 Main Mall \\
  Vancouver, BC, Canada, V6T 1Z4
  \item USDA
  \end{enumerate}
}

\def\runninghead{}

\def\corr{\noindent
  $\star$ Corresponding author: jenna.melanson@ubc.ca \\
  }

\newcommand{\mstitlepage}{
  \paragraph{Running head:} \textsc{\runninghead}
  \parindent=0pt
  \begin{center}
    {\LARGE \title \par}
    \vskip 3em
    {\large
      \lineskip .75em
      \begin{tabular}[t]{c}
        \author
      \end{tabular}\par}
    \vskip 1.5em
  \end{center}\par
  \affiliation
  \corr
  \extras
}

\begin{document}
\mstitlepage
\doublespacing
\linenumbers
\clearpage
\begin{abstract} 
abstract text
\end{abstract} 

\textbf{Keywords:} \keywords
\clearpage
\section{Introduction}
\cite{gelmanRsquaredBayesianRegression2019}
\section{Methods}

\subsection{Study system}
Our study took place in the Lower Fraser Valley in southwestern British Columbia, Canada, in an agricultural system dominated by mixed vegetable, hay, and perennial berry production. From 1984-2018 the Lower Fraser Valley underwent a 13\% decrease in forest patch area, mainly resulting from conversion to urban or agricultural land use \parencite{paulTrackingChangesSoil2020a}. .... \emph{decide what else needs to be said in this section later on!}

Field surveys were conducted across six replicate landscapes distributed throughout the Lower Fraser Valley. Each landscape encompassed roughly 3 sq km of farmland interspersed with rural/suburban residence. Landscapes were initially chosen to span a gradient of configurational and composition diversity metrics, including Shannon's diversity, edge density, and the ratio of annual to perennial crop cultivation. 

Thirty sampling transects (50 meters x 2 meters) were established in each landscape, spaced as evenly as possible based on land-access and the availability of foraging resources on which to observe bees. We did not survey in active crop fields except for high-bush blueberry, which offers floral resources during its spring bloom and can sometimes host other flowers such as clovers (\textit{Trifolium spp.}) and flatweed (\textit{Hypochaeris radicata}) later in the season. A total of \emph{INSERT TOTAL TRANSECT NUMBER HERE} were surveyed over the course of two years (2022-2023) due to changes in land access within and between years. 

\subsection{\textit{Bombus} collections and floral surveys}
Each landscape was surveyed during 10 sampling rounds in year one (May-August 2022) and 17 sampling rounds in year two (March-August 2023). During each round of sampling, surveys were conducted on \emph{INSERT MEAN PLUS OR MINUS SD OF SAMPLING EFFORT} (mean \pm SE) transects. 

\textit{Bombus} surveys at each transect entailed 5 minutes of active search time (totaling 140 hours in 2022 and \emph{INSERT TOTAL SAMPLING EFFORT 2023} in 2023), during which the stopwatch was paused whenever a foraging bumble bee was sighted. Specimens were captured by netting, placed into sterile 15 mL tubes, and immediately placed on ice before transfer to a -80\degree C freezer at the end of the day.  Surveys were conducted on days when the temperature was above 12\degree C (10\degree C for queen surveys) and wind speeds below 2.5 m/s. In 2022, all \texit{Bombus} species were collected; in 2023 only the focal species (\textit{B. mixtus} and \textit{B. impatiens} were collected.

To assess floral quality, all flowering plants within the transect area were identified to species or genus level. Abundance estimates were given for each species using a log-scale (i.e., 0 = 1-10 inflorescences, 1 = 11-100 inflorescences, 2 = 101-1000 inflorescences, 3 = 1001-10,000 inflorescences,  4 = 10,000+ inflorescences). Floral survey data was later filtered to exclude species which bumble bees were never observed visiting (based on over ~3,400 visitation events in 2022, and ~3,500 visitation events in 2023). This filtering step was included to reduce the noise introduced by a variety of herbaceous weeds with flowers too small to attract or support bumble bee foragers, but which were frequently observed on the transects in high abundance. 


\emph{START EDITS FROM HERE}
\subsection{Landscape analyses}
Land cover maps were developed for each study site based on manual classification of Google Earth satellite imagery (2021) and site visits. Briefly, land cover was classified into 16 categories: annual row crops, blueberry, cranberry, other perennials, polyculture, hay meadows, pasture, fallow, grassy field margins, hedgerows (tree-dominated), hedgerows (blackberry-dominated), forest, wetlands, urban/suburban, roads, and water). These land cover types were chosen based on their hypothesized provisioning of nesting/floral resources and differences in disturbance regimes (see Table \ref{tab:landscapetable} for details). Land cover was mapped at 2-meter resolution in QGIS \parencite{QGIS_software}. Landscape Shannon diversity, the proportion of each landscape in blueberry cultivation, and the proportion of each landscape consisting of edge habitat (grassy field margins and hedgerows) were calculated within a 500-meter buffer surrounding each transect using the package \textit{lsm} \parencite{lsm-package}. Here, "edge density" is defined as the proportion of area within each buffer zone that was classified as a hedgerow or grassy margin. A buffer size was 500m was chosen based on the average foraging distance observed for bumble bee species across many studies \parencite{molaReview2025}; we used average foraging distance (as opposed to maximum foraging distance) as it represents the area accessible to the majority of individuals surveyed at a location, including those from species with smaller foraging ranges.


\section{Results}
\section{Discussion}

\bibliographystyle{wileyNJD-Harvard.bst}
  \bibliography{refs.bib}
  \setlength{\parskip}{1em}
\clearpage


%\section*{Acknowledgements}

%\section*{Conflict of Interest}
%You may be asked to provide a conflict of interest statement during the submission process. Please check the journal's author guidelines for details on what to include in this section. Please ensure you liaise with all co-authors to confirm agreement with the final statement.

\section*{Supporting Information}

\setcounter{figure}{0}
\setcounter{table}{0}


%\input{supplement}

\end{document}
