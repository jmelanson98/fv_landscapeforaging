\documentclass[11pt]{article}
\usepackage[a4paper, margin=1in]{geometry}
\usepackage{graphicx}
\setlength{\parindent}{0pt}
\setlength{\parskip}{0.8em}


\def\title{Bumble bee colony density and foraging distance in relation to agricultural landscape simplification}
\def\extras{Entomological Society Meeting -- Portland, 2025}
\def\author{Jenna B. Melanson$^{1*}$, Tyler T. Kelly$^{2}$, Natalia Clermont$^{2}$, \\
Jonathan B. Koch$^{3}$, Claire Kremen$^{1,2}$}

\def\affiliation{
  \begin{enumerate}
  \item Biodiversity Research Center\\
  Department of Zoology \\
  University of British Columbia \\
  2212 Main Mall \\
  Vancouver, BC, Canada, V6T 1Z4
  \item Institute for Resources, Environment, and Sustainability \\
  University of British Columbia \\
  2202 Main Mall \\
  Vancouver, BC, Canada, V6T 1Z4
  \item USDA
  \end{enumerate}
}
\def\corr{\noindent
  $\star$ Presenting author: jenna.melanson@ubc.ca \\
  }

\newcommand{\mstitlepage}{
  \parindent=0pt
  \begin{center}
    {\LARGE \title \par}
    \vskip 1em
    {\large \extras \par}
    \vskip 1em
    {\large
      \lineskip .75em
      \begin{tabular}[t]{c}
        \author
      \end{tabular}\par}
    \vskip 1.5em
  \end{center}\par
  \affiliation
  \corr
}

\begin{document}
\mstitlepage
\section{Abstract}
%Habitat configuration and quality can shape animal movement through multiple mechanisms. Functional connectivity of habitat patches may facilitate or impede movement ability, while the spatial distribution of resources may influence the frequency or length of foraging bouts. 
Agricultural landscapes represent a mosaic of disturbance regimes, where land use types vary in their ability to provision nesting and floral resources for pollinators and other insects. For highly-mobile, central-place foragers like bumble bees, resource distribution is believed to be a major driver of foraging distance, duration, and profitability. Land cover configuration and quality may influence both foraging behaviour and effective population size, with repercussions for crop pollination and bumble bee conservation. Landscape-simplification (e.g., increased field sizes, reduced crop diversity, and removal of non-crop vegetation along field margins) may either increase foraging ranges to accomodate colony nutritional demands, or reduce foraging ranges if structural connectivity is key in facilitating bumble bee movement. To test these hypotheses, we captured and microsatellite-genotyped bumble bee workers of two species (\textit{Bombus mixtus} and \textit{B. impatiens}) in six replicate landscapes across an agricultural region of southwestern British Columbia, Canada. We will fit spatially explicit genetic capture-recapture models to assess the effects of two landscape metrics (field margin density and patch-type interspersion) on colony abundance, lineage turnover, and foraging distance of each species. Additionally, we will make comparisons between a common but understudied native species (\textit{B. mixtus}) and its invasive relative (\textit{B. impatiens}) to determine whether differential responses to landscape characteristics may be facilitating the rapid spread of \textit{B. impatiens} outside its native range. Our work will provide insights for pollinator movement ecology in anthropogenically disturbed landscapes, while also deepening our understanding of how these systems may influence species invasions.


\end{document}
