\documentclass[11pt]{article}
\usepackage[a4paper, margin=1in]{geometry}
\usepackage{graphicx}
\setlength{\parindent}{0pt}
\setlength{\parskip}{0.8em}


\def\title{Bumble bee colony density and foraging distance in relation to agricultural landscape simplification}
\def\extras{Entomological Society Meeting -- Portland, 2025}
\def\author{Jenna B. Melanson$^{1*}$, Tyler T. Kelly$^{2}$, Natalia Clermont$^{2}$, \\
Jonathan B. Koch$^{3}$, Claire Kremen$^{1,2}$}

\def\affiliation{
  \begin{enumerate}
  \item Biodiversity Research Center\\
  Department of Zoology \\
  University of British Columbia \\
  2212 Main Mall \\
  Vancouver, BC, Canada, V6T 1Z4
  \item Institute for Resources, Environment, and Sustainability \\
  University of British Columbia \\
  2202 Main Mall \\
  Vancouver, BC, Canada, V6T 1Z4
  \item USDA
  \end{enumerate}
}
\def\corr{\noindent
  $\star$ Presenting author: jenna.melanson@ubc.ca \\
  }

\newcommand{\mstitlepage}{
  \parindent=0pt
  \begin{center}
    {\LARGE \title \par}
    \vskip 1em
    {\large \extras \par}
    \vskip 1em
    {\large
      \lineskip .75em
      \begin{tabular}[t]{c}
        \author
      \end{tabular}\par}
    \vskip 1.5em
  \end{center}\par
  \affiliation
  \corr
}

\begin{document}
\mstitlepage
\section{Abstract}

Agricultural landscapes present a mosaic of disturbance regimes, where land use types vary in their ability to provision nesting and floral resources for pollinators and other insects. For central-place foragers like bumblebees, spatial adjacency of sufficient quantities of nesting and floral resources is key for maintaining sufficient colony-level energy returns to support growth and reproduction. However, even closely related species are known to vary in both nesting habitat preferences and foraging behaviours, including the length of foraging bouts and the tendency to locate and aggregate on highly abundant floral resource patches. Differences in foraging distance have even been suggested as one driver of coexistence for species which occupy otherwise similar niche spaces. Agricultural landscape simplification, including reduction in seminatural habitat area, decreased crop diversity, and increased field sizes, may upset this balance, potentially favouring the species most adapted to profit from spatiotemporally sparse but highly abundant mass-flowering crops. These differing traits may even drive species range expansion/invasion (e.g., \emph{Bombus terrestris} in Europe and globally, \emph{B. impatiens} in North America). To test for differences between native and introduced species' responses to landscape simplification, we performed queen nest searching surveys, quantified colony densities, and fit spatially explicit models of foraging distance (including the effects of floral abundance on patch visitation) for two species (\textit{B. mixtus} and \textit{B. impatiens}) in six replicate landscapes across an agricultural region of southwestern British Columbia, Canada. While our previous work shows that the abundance of native \emph{B. mixtus} workers shows a stronger decline in response to landscape simplification than \emph{B. impatiens}, here we seek to determine whether these effects result from (i) different preferences for nesting habitat or (ii) differences in foraging distance between species. Our work will provide insights for pollinator movement and habitat use in anthropogenically disturbed landscapes, while also deepening our understanding of how these systems may influence species invasions.


\end{document}
